\documentclass[fontsize=11pt]{article}
\usepackage{amsmath}
\usepackage{latexsym}
\usepackage{amsfonts}
\usepackage[normalem]{ulem}
\usepackage{soul}
\usepackage{array}
\usepackage{amssymb}
\usepackage{extarrows}
\usepackage{graphicx}
\usepackage[backend=biber,
style=numeric,
sorting=none,
isbn=false,
doi=false,
url=false,
]{biblatex}\addbibresource{bibliography.bib}

\usepackage{subfig}
\usepackage{wrapfig}
\usepackage{wasysym}
\usepackage{enumitem}
\usepackage{adjustbox}
\usepackage{ragged2e}
\usepackage[svgnames,table]{xcolor}
\usepackage{tikz}
\usepackage{longtable}
\usepackage{changepage}
\usepackage{setspace}
\usepackage{hhline}
\usepackage{multicol}
\usepackage{tabto}
\usepackage{float}
\usepackage{multirow}
\usepackage{makecell}
\usepackage{fancyhdr}
\usepackage[toc,page]{appendix}
\usepackage[hidelinks]{hyperref}
\usetikzlibrary{shapes.symbols,shapes.geometric,shadows,arrows.meta}
\tikzset{&gt;={Latex[width=1.5mm,length=2mm]}}
\usepackage{flowchart}\usepackage[paperheight=11.0in,paperwidth=8.5in,left=1.0in,right=1.0in,top=1.0in,bottom=1.0in,headheight=1in]{geometry}
\usepackage[utf8]{inputenc}
\usepackage[T1]{fontenc}

\title{CSC110 Project Proposal: \\
How Will Stock Markets ‘Weather’ Climate Change?}
\author{Divit Singh, Praket Kanaujia, Marton Kovacs,  Nimit Bhanshali}
\date{Friday, November 6, 2020}

\begin{document}
\maketitle

\section*{Problem Description and Research Question}
Capitalism and industrial processes are often associated as one of the
culprits of climate change, but what is often left unnoticed is the fact
that this very phenomenon may impact economies and stock market indexes
(for the better or the worse). This is because real life events can have
a severe impact on stock market fluctuations as these events can have a direct impact on the future of many stocks and this, in turn leads consumers to respond to these predictions (1). Our goal is to understand how companies and countries contribute to climate change, measured by annual highest temperature, by looking at factors such as their stock market indexes as well as fossil fuel consumption.\\

\noindent \textbf{Research Question:}\\

\textbf{How are stock market indexes of companies and countries affected by initiatives made to tackle climate change, indicated by fossil fuel consumption?}\\

\textbf{How do companies and countries contribute to climate change as indicated by their stock market indexes and fossil fuel consumption, respectively?}\\

\noindent We plan to categorise nations into three groups:\

\begin{enumerate}
    \item[1.] High Stake: Countries that are highly impacted by climate change and hence have a stake in working towards reducing emissions, characterised by a decrease in fossil fuel consumption.
    \item[2.] Medium Stake: Countries that do not necessarily have immediate danger from climate change, but do have the resources to bring out change and thus their fossil fuel consumption is fairly steady.
    \item[3.] Low Stake: Countries that have little to no incentive towards reducing the impact of climate change and their fossil fuel consumption is increasing.
\end{enumerate}

\noindent In a similar vein, we shall identify companies from our chosen countries that belong to one of the following criteria:\

\begin{enumerate}
    \item[1.] Green Companies: Firms that are developing products that have a direct / significant role in reducing emission levels.
    \item[2.] White Companies: Companies that do not have a direct impact on  climate change, either by working to produce greener alternatives or by directly impacting emission levels.
    \item[3.] Red Companies: Firms that profit from the sale of fossil fuels or release high levels of green gas emission in their production processes.
\end{enumerate}

It is important to understand the situations and ideologies of nations
and corporations and hence the reasons for their approach towards
climate change. By doing so, it will be possible to plan out effective
strategies to ensure that economic and environmental interests are
secured in our bid to improve the worsening state of global climates.
This project shall assess relevant data, as well as produce a model to
analyze the trends in stock market indices of nations and firms to better understand their role in combating climate change. \\


\section*{Data set Description}
\textbf{‘Daily Temperature of Major Cities’}\\

A data set we will be using throughout the course of our project is the ‘Daily Temperature of Major Cities’ data set found on Kaggle (2). This data set contains the recorded daily mean temperature in 324 major cities around the globe starting from January 1, 1995 all the way up till April 4, 2020. We will be specifically focusing on the daily mean temperatures across various cities in selected countries. Using Python, we will calculate the highest annual temperature for each of the selected countries to use in our program. All this data is packaged in a file named ‘city$\_$temperature.csv’ and has the file-type ‘.csv’. \\

\noindent \textbf{‘Fossil Fuel Consumption’}\\

Another data set we will be using as the dependent variable throughout
this project is the 'Fossil Fuel Consumption' data set found on OurWorldInData (3). This data set contains the yearly consumption of fossil fuels, in TeraWatt hours, for each country from 1965 to 2019. We will be specifically focusing on the yearly fossil fuel consumption for selected countries. This information will be used to plot a graph of fossil fuel consumption vs annual highest temperature. This data is available in a file named 'fossil-fuel-primary-energy.csv' and has the filetype '.csv'.\\

\noindent \textbf{'Stock Market Index'}\\

The third data set we will be using as another dependent variable
throughout the project is the 'Stock Market Index'. For each of the two
countries in our analysis and 3 companies per country, there will be individual data sets regarding their respective stock market index. These data sets are found on YahooFinance (4) and, depending on the country and company, the start date varies. Due to the large size of the data set, we will only be looking at the close prices per month. Additionally, we will also be taking the average of all the months to find the yearly data as
all the other data is being looked at annually. The data file is
different for different companies however, all the files have the type
'.csv'. \\

\noindent The stock market indices of the 2 counties and their respective 3 companies we will be looking at are:
\begin{enumerate}
    \item Japan
        \begin{enumerate}
            \item Tokyo Electric Power Company (TEPCO)
            \item Tokyo Gas
            \item Mitsubishi
        \end{enumerate}
    \item India
        \begin{enumerate}
            \item Hindustan Unilever
            \item Indian Oil
            \item Tata Chemicals
        \end{enumerate}
\end{enumerate}\\

We decided to choose the Japan and India as we wanted to look examine a sample with varying economic status and understand their response to the global issue of climate change. Within these nations, we chose to examine companies from varying industries to understand the roles each company plays in combating climate change.\\

\section*{Computational Overview}

\begin{enumerate}
\item Conversion Module

This module utilizes the Python library, Pandas (5), to read data sets that are in .csv format and convert them into data frames. Then, the functions in this module execute Pandas functions to filter out specific columns and calculations such as max() and mean(). Finally, each of these data frames are converted into dictionaries to allow for further computations. This module contains three functions that manipulates each of the three types of data sets (Daily Temperature of Major Cities, Stock Market Index, and Fossil Fuel Consumption) in the manner stated above.

\item Graphing Module

This module creates line graphs using Plotly (6) for both countries
and companies. The country graph uses the dictionaries from the previous module to plot graphs of Fossil Fuel Consumption and Annual Highest Temperature against Year for users to identify trends in climate change and the nation's response to it. For the company graph, Mean Annual Stock Value and Annual Highest Temperature are plotted against Year to understand how the organisations' valuation has fared as temperatures changed in the country where it is operating. These are multi-variable graphs with 2 y-axis values in order to better illustrate the variance in the aforementioned factors.

Next, the country or company is analysed and it is concluded to belong
to the criteria elucidated prior in the document. The result is output
in the console.


\item Main

This module is what the user will be utilising to employ the software
for analysis of countries and companies.

In order to undertake a country analysis, one has to input the name of
the country to be analysed and the start, end years of the analysis. In
the case of our data sets, this is limited to 1995-2019 as this is the
largest time period common to all of the data sets used in the analysis.

Similarly, a company analysis requires the company name, country of
operation name, and start, end years of the analysis. Results will be
viewed via graph as well as a console message that informs the user of
the conclusion of the automated analysis.

\end{enumerate}

\section*{Instructions}

Below are the links to all the raw data sets that we have used:

\begin{enumerate}

    \item
    Temperature data set - \href{https://www.kaggle.com/sudalairajkumar/daily-temperature-of-major-cities}{https://www.kaggle.com/sudalairajkumar/daily-temperature-of-major-cities}\par

    \item Fossil Fuel Consumption data set - \url{https://ourworldindata.org/grapher/fossil-fuel-primary-energy}\par

\item Stock Indexes data set -
\begin{enumerate}
    \item Tokyo Electric Power Company - \url{https://finance.yahoo.com/quote/9501.T/history?period1=946944000&period2=1607990400&interval=1mo&filter=history&frequency=1mo&includeAdjustedClose=true}\par

    \item Tokyo Gas - \url{https://finance.yahoo.com/quote/9531.T/history?period1=946944000\&period2=1607817600\&interval=1mo\&filter=history\&frequency=1mo\&includeAdjustedClose=true}\par

    \item
    Mitsubishi - \url{https://finance.yahoo.com/quote/8058.T/history?period1=946944000&period2=1607904000&interval=1mo&filter=history&frequency=1mo&includeAdjustedClose=true}\par

    \item Hindustan Unilever - \url{https://finance.yahoo.com/quote/HINDUNILVR.NS/history?period1=820454400\&period2=1607817600\&interval=1mo\&filter=history\&frequency=1mo\&includeAdjustedClose=true}\par

    \item
    Indian Oil - \url{https://finance.yahoo.com/quote/IOC.BO/history?period1=807926400\&period2=1607817600\&interval=1mo\&filter=history\&frequency=1mo\&includeAdjustedClose=true}\par

    \item
    Tata Chemicals - \url{https://finance.yahoo.com/quote/TATACHEM.BO/history?period1=663206400\&period2=1607817600\&interval=1mo\&filter=history\&frequency=1mo\&includeAdjustedClose=true}\par

\end{enumerate}
\end{enumerate}

The processed data set is available through send.utoronto.ca, with the Claim ID and Claim Passcode listed below: \\

\noindent Processed Data set: \\
    send.utoronto.ca: \\
    Claim ID: FboEVRccoNnApeHD \\
    Claim Passcode: m9pimaWfTnQNeJtD \\

Please save these data sets in the same folder as the Python files (no sub-folders). Also, please mark the directory in which all the Python files and the data sets are saved in as Source Root.\\

On the next page, we have provided sample results when main.py is run. In summary, when main.py is run, an interactive interface will be displayed where the user will be instructed to input a choice from the given options. Following these choices, the chosen function will automatically be called and present the output as shown below. \\

\pagebreak

Using these data sets, we performed the python function, that we created based on the computational plan mentioned above, for Tokyo Gas and Hindustan Unilever, the results of which can be seen in the screenshots below. The function can be performed by running main.py and by answering the prompts, as shown below in the console screenshots. The console return an output about whether the company is Green, Red or White based on the stocks, the fossil fuel consumption and the highest annual temperature. An interactive Plotly graph is also part of the output, as seen in the graph images below. In this graph, it is possible to not only see the trend but also the values inputted to get the points on the graph.\\
\begin{center}
\includegraphics[width = 15cm]{TokyoGas Console.png}
\includegraphics[width = 15cm]{TokyoGas Graph.png}
\includegraphics[width = 15cm]{HindustanUnilever Console.png}
\includegraphics[width = 15cm]{HindustanUnilever Graph.png}

\end{center}


\section*{Changes To Plan}
In order to separate climate change as an economic force from all the
other economic forces as well as help to classify the countries and
companies into the categories mentioned above, we decided to look at
fossil fuel consumption as well and form a correlation between the
consumption and stocks. Fossil fuel consumption is a better measure than
expenditure as the latter is prone to inflation which will artificially
raise the value. Additionally, after feedback from the TA, we chose to
look at the highest temperature of the years instead of the average
temperature as it would require us to take the average of averages which
would reduce accuracy. Moreover, we decided against using the Scrapy
library as the data being outputted was not in the format that we could
work with and thus, we instead chose to find existing .csv files for the
data that we required.\\

\section*{Discussion Section}

\ \ \ \ The results of the computational exploration does help us answer
the posed questions as the resulting graph and program itself allows us
to form a correlation between the fluctuations in the market stocks with
regard to climate change. Through this data, we are able to group the
countries as well as the companies based on their commitment to climate
change. \\

We encountered a few limitations and obstacles along the course of this
program. The first was with regard to Scrapy. Due to the large time
range that we were looking at, we could not use Scrapy to output stock
market data as the source we were using, Yahoo Finance, had an infinite
scrolling mechanism that Scrapy was not able to trigger. Moreover, we
had to find a different, downloadable source for our fossil fuel data as
when trying to use Scrapy on the first source, Pandas was unable to read
the output file. Another limitation we encountered was with the plotting
of the graph. Due to the varied fluctuations of the data being used, we
could not plot a direct temperature vs fossil fuel consumption graph or
a temperature vs stock value index graph. Thus, we instead opted to have
an indirect approach with having 2 line graphs on one graph with the x
axis being the time in years. This allows us to find a correlation based
on the net movement of the line curves. \\

While this program is able to look at how fossil fuel affects the stock
market of countries, it is not able to downplay the other factors that
affect stock market changes. Thus, if this program can be paired with
other programs that are able to remove the effect of external factors on
the stock market, we can use this program to accurately group the
countries as well as the companies based on their commitment to climate
change. \\

To elaborate, when the software suggests that a given country is 'High
Stake', it would imply that one can expect to see further reduction in
fossil fuel consumption as well as increased measures to combat climate
change, as long as the government maintains its current standing on
climate change. On the other hand, a 'Low Stake' country would be one
that does not appear to work towards reducing fuel dependency, so it
becomes clear that such a nation is either not impacted by climate
change and may in fact even increase fossil fuel consumption in order to
increase production capacity. \\

Similarly, when a company is concluded to be a 'Red Company', we can
assume that it has significant dependency or is an explicit provider of
fossil fuels, and hence does not benefit from a rise in climate change
initiatives. Conversely, a 'Green Company' benefits from reduced fossil
fuel consumption as that implies an increased dependency on alternative,
renewable resources. \\

Further, a country that is 'Medium Stake' or a 'White Company' does not
seem to be impacted much by climate change or fossil fuel consumption
respectively. \\

Unfortunately, in some cases, the wildly fluctuating data prevents an
informed analysis from taking place, and our program is unable to make
an accurate conclusion. Ignoring these cases, the user is able to gather
information about the policies of governments and corporate
organisations and hence make more informed decisions when presenting
climate changes to these groups. As a result, one is likely to present a
more compelling argument that will meet the requirements of both nature
and economy. \\

As a further exploration, we would like to expand our program so that it is compatible with a larger range of data sets including stock market indices for a larger number of countries and companies. Additionally, another further exploration would be to account for a greater number of factors that can influence the stock market indices in addition to fossil fuel consumption. Some examples of such could be the passing of major bills and laws that impact climate change, the distribution of energy amongst renewable and non-renewable resources, etc. By accounting for a larger number of such factors, we will be able to have a more accurate and clear understanding of the stance that each of these nations and companies have in response to climate change. Such information will be key to determine what actions are necessary by large governing bodies such as the United Nations, to either sanction or support certain companies and nations to successfully tackle the issue of climate change.\\

\vspace{\baselineskip}
\section*{References}\par


\begin{enumerate}

    \item[(1)] Beers, B. (2020, August 28). How the News Affects Stock Prices. Retrieved December 13, 2020, from \url{https://www.investopedia.com/ask/answers/155.asp}

    \item[(2)] Earth, B. (2017, May 01). Climate Change: Earth Surface
    Temperature Data. Retrieved December 13, 2020, from \url{https://www.kaggle.com/berkeleyearth/climate-change-earth-surface-temperature-data}

    \item[(3)] Ritchie, H., \& Roser, M. (2017, October 02). Fossil Fuels. Retrieved December 13, 2020, from \url{https://ourworldindata.org/fossil-fuels}

    \item[(4)] Serwer, A. (Ed.). (1997, January 19). Yahoo Finance - Stock Market Live, Quotes, Business \& Finance News. Retrieved December 14, 2020, from \url{https://finance.yahoo.com/}

    \item[(5)] Management, A. (2008, January 11). Pandas. Retrieved December 13, 2020, from \url{https://pandas.pydata.org/}

    \item[(6)] Johnson, A., Parmer, J., Parmer, C., \& Sundquist, M. (2013, December 25). The front end for ML and data science models. Retrieved December 14, 2020, from \url{https://plotly.com/}

\end{enumerate}
\end{document}
